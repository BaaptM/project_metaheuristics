\documentclass[a4paper]{article}
\usepackage[french]{babel}
\usepackage[utf8]{inputenc}
\usepackage[T1]{fontenc}
\usepackage{hyperref}
\usepackage{graphicx}
\usepackage{appendix}
\usepackage{caption}
\usepackage{pdfpages}
\usepackage{mathtools}
\usepackage{cite}
\usepackage{array}
\usepackage{multirow}
\usepackage{listings}
\lstset{language=Python}
\begin{document}

\begin{titlepage}
\newcommand{\HRule}{\rule{\linewidth}{0.5mm}} 
\center 
 
%----------------------------------------------------------------------------------------
%	HEADING SECTIONS
%----------------------------------------------------------------------------------------

\textsc{\LARGE Université de Bordeaux}\\[1.5cm] 
\textsc{\Large Projet Métaheuristique}\\[0.5cm] 
\textsc{\large M1 Informatique}\\[0.5cm] 

%----------------------------------------------------------------------------------------
%	TITLE SECTION
%----------------------------------------------------------------------------------------

\HRule \\[0.4cm]
{ \huge \bfseries Partitionnement de graphe en classes}\\[0.4cm] % Title of your document
\HRule \\[1.5cm]
 
%----------------------------------------------------------------------------------------
%	AUTHOR SECTION
%----------------------------------------------------------------------------------------

\begin{minipage}{0.4\textwidth}
\begin{flushleft} \large
\emph{Auteurs : }\\
Norbert \textsc{Feron},\\
Baptiste \textsc{Masset}
\end{flushleft}
\end{minipage}
~
\begin{minipage}{0.4\textwidth}
\begin{flushright} \large
\emph{Encadrant : } \\
Marc Michel \textsc{Corsini} 
\end{flushright}
\end{minipage}\\[2cm]

%----------------------------------------------------------------------------------------
%	DATE SECTION
%----------------------------------------------------------------------------------------

{\large \today}\\[2cm]

%----------------------------------------------------------------------------------------
%	LOGO SECTION
%----------------------------------------------------------------------------------------

\includegraphics{img/logo.png}\\[1cm]
 
%----------------------------------------------------------------------------------------

\vfill
\end{titlepage}
%----------------------------------------------------------------------------------------

\tableofcontents

\newpage
\section{Introduction}

Dans le cadre du cours de recherche opérationnel de master 1 informatique a l'université de Bordeaux, ce projet a pour but de mettre en pratique des algorithmes et méta-algorithme, tel que la descente de gradient, le Recuit simulé ou encore la méthode Tabou. Pour cela nous disposons de graphes pondérés de différentes taille (de 5-7 sommets a plus de milles), sur lesquelles nous allons tester et comparé les résultats de nos différents algorithme appliqué au problème de partitionnement de graphe en classes à peu près équitables.

\subsection{Le problème}
Notre problème: partitionner un graphe en K classes en minimisant le poids interclasses. Cela consistes à placer chaque sommet du graphe dans une des classe de manières à réduire le somme du poids des arrêtes n'appartenant pas à la même classe.

\subsection{Nombre de classes}
Nous avons choisi de laisser le nombre de classes libre, l'utilisateur lance donc les algorithmes en donnant comme entrée un graphe pondéré et un nombre de classe. 

\subsection{Définition de l'équitable}

Nous avons choisi de ne pas implémenter la notion  d'équilibre entre les classes dans nos méta-algorithmes, l'utilisateur dispose d'un variable supplémentaire appelé delta\_max; elle correspond à l'écart maximum entre les classes pour qu'une solution soit considérée comme valide et puisse être retournée en sortie des différents algorithmes utilisés dans ce projet.


\section{Méthode d’énumération}


\section{Descente de Gradient}

\section{Recuit Simulé}

\section{Méthode Tabou}

\section{Comparaison des résultats}

\section{Conclusion}

\end{document}