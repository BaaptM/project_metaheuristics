\documentclass[a4paper]{article}
\usepackage[french]{babel}
\usepackage[utf8]{inputenc}
\usepackage[T1]{fontenc}
\usepackage{hyperref}
\usepackage{graphicx}
\usepackage{appendix}
\usepackage{caption}
\usepackage{pdfpages}
\usepackage{mathtools}
\usepackage{cite}
\usepackage{array}
\usepackage{multirow}
\usepackage{listings}
\lstset{language=Python}
\begin{document}

\begin{titlepage}
\newcommand{\HRule}{\rule{\linewidth}{0.5mm}} 
\center 
 
%----------------------------------------------------------------------------------------
%	HEADING SECTIONS
%----------------------------------------------------------------------------------------

\textsc{\LARGE Université de Bordeaux}\\[1.5cm] 
\textsc{\Large Projet Métaheuristique}\\[0.5cm] 
\textsc{\large M1 Informatique}\\[0.5cm] 

%----------------------------------------------------------------------------------------
%	TITLE SECTION
%----------------------------------------------------------------------------------------

\HRule \\[0.4cm]
{ \huge \bfseries Partitionnement de graphe en K classes}\\[0.4cm] % Title of your document
\HRule \\[1.5cm]
 
%----------------------------------------------------------------------------------------
%	AUTHOR SECTION
%----------------------------------------------------------------------------------------

\begin{minipage}{0.4\textwidth}
\begin{flushleft} \large
\emph{Auteurs : }\\
Norbert \textsc{Feron},\\
Baptiste \textsc{Masset}
\end{flushleft}
\end{minipage}
~
\begin{minipage}{0.4\textwidth}
\begin{flushright} \large
\emph{Encadrant : } \\
Marc Michel \textsc{Corsini} 
\end{flushright}
\end{minipage}\\[2cm]

%----------------------------------------------------------------------------------------
%	DATE SECTION
%----------------------------------------------------------------------------------------

{\large \today}\\[2cm]

%----------------------------------------------------------------------------------------
%	LOGO SECTION
%----------------------------------------------------------------------------------------

\includegraphics{img/logo.png}\\[1cm]
 
%----------------------------------------------------------------------------------------

\vfill
\end{titlepage}
%----------------------------------------------------------------------------------------

\tableofcontents

\newpage

\section{Introduction}
Dans le cadre de l'UE Graphes et Recherche Opérationnel du Master 1 Informatique de l'Université de Bordeaux, nous avons mis en œuvre ce projet qui a pour but de mettre en pratique des heuristiques et méta-heuristiques, tels que la descente de gradient, le recuit simulé ou encore la méthode tabou, sur un problème de partitionnement de graphe. Pour cela nous disposons de graphes pondérés, de différentes tailles (de cinq sommets à plus de milles), sur lesquelles nous allons tester et comparer les résultats de nos différents algorithmes appliqués au problème de partitionnement de graphe en K classes ``à peu près équitables''.

\subsection{Définition d'une méta-heuristique}
\begin{textit}
Une méta-heuristique est un algorithme d’optimisation visant à résoudre des problèmes d’optimisation difficile (souvent issus des domaines de la recherche opérationnelle, de l'ingénierie ou de l'intelligence artificielle) pour lesquels on ne connaît pas de méthode classique plus efficace.
\end{textit}

Les méta-heuristiques sont généralement des algorithmes stochastiques itératifs, qui progressent vers un optimum global, c'est-à-dire l'extremum global d'une fonction, par échantillonnage d’une fonction objectif. Elles se comportent comme des algorithmes de recherche, tentant d’apprendre les caractéristiques d’un problème afin d’en trouver une approximation de la meilleure solution.

\subsection{Problématique}
Notre problème est le suivant : partitionner un graphe en K classes ``à peu près équitable'' tout en minimisant le poids des arêtes interclasses. En d'autres termes, cela consiste à placer chaque sommet du graphe dans une des classe de manières à minimiser la somme des poids des arrêtes n'appartenant pas à la même classe.
\begin{center}
min($\sum\limits_{i} \sum\limits_{j}  \omega (i,j)$)
\end{center}
avec $i \in K1, j \in K2, i \ne j$ et $\omega (i,j)$ le poids entre $i$ et $j$.

\subsection{Nombre de classes}
En ce qui concerne le nombre de classes en lequel vont être partitionnés nos graphes, nous avons choisi de laisser ce paramètre externe au problème. Autrement dit, c'est à l'utilisateur de définir ce paramètre à l’exécution suivant le nombre de sommets du graphe et ses attentes.
Aussi le nombre de solutions potentielles vaut $k^n$ avec $k$ le nombre de classes et $n$ le nombre de sommets du graphe. 

Attention donc à ne pas choisir un $k$ trop grand pour un graphe comportant beaucoup de sommets au risque de ne pas avoir de retour du programme en un temps raisonnable. 

\subsection{Définition de l'équité}

Pour ce qui est de la notion d' ``à peu près équitables'' entre les classes, nous avons choisi de laisser ce paramètre à l'appréciation de l'utilisateur par soucis de généricité. Celui-ci dispose d'une variable supplémentaire appelée ``delta\_max''. Cette variable correspond à l'écart maximum d'éléments autorisé entre les classes pour qu'une solution soit considérée comme acceptable et puisse être retournée en sortie des différents algorithmes utilisés dans ce projet.

De plus, le score des solutions est pénalisé par la différence du nombre de sommets dans les classes par une constante appelé $\mu$ ``mu'' $\in [0,1]$. Ainsi, plus la disparité entre les classes sera grande plus la pénalité sera élevée.

\subsection{Fonction objectif}
Cette fonction permet de décrire la qualité d'une solution.


L'objectif du projet est de minimiser la somme du poids des arrêtes interclasses, nous retenons donc les solutions dont le score est le plus faible. Le calcul du score est : 
\begin{center}
$\sum\limits_{i} \sum\limits_{j}  \omega (i,j)) + \mu * c(\delta)$
\end{center}
avec $i \in K1, j \in K2, i \ne j$, $\omega (i,j)$ le poids entre $i$ et $j$, $\mu$ une constante $\in [0,1]$, $\delta$ l'écart en terme d'éléments entre K1 et K2 et $c(\delta)$ le rapport entre $\delta$ et le nombre total de sommets.

\section{L'espace de recherche}
Lors de l'utilisation des méta-heuristiques, l'ensemble des solutions possibles forment un espace de recherche borné. Le but de nos méta-algorithme est donc de trouver l'optimum, la meilleure solution au problème.
	\subsection{Notion de voisinage}
	\subsection{Mouvement élémentaire}

\section{Méthode d’énumération}
La solution la plus simple consiste à énumérer toutes les solutions valides et de calculer la somme du poids des arrêtes interclasses, les solutions avec le meilleur score sont retournées en résultats de l'algorithme d'énumération. Les solutions retournées sont donc toujours les solutions optimales, cependant la puissance de nos calculateurs fait défaut et en dépassant un certain nombre de sommets, nos ordinateurs ne disposent pas d'un puissance de calcul suffisante et l'énumération des solutions devient impossible à effectuer en un temps raisonnable.



\section{Descente de Gradient}

\section{Recuit Simulé}

\section{Méthode Tabou}

\section{Test \& Résultats}

Nous avons testé nos algorithmes sur des graphes allant de 4 sommets à 100 sommets, les méta-algorithmes de descente de gradient, recuit simulé et méthode tabou sont lancé de façon indépendante en parallèle, après un nombre d'itérations (fixé à 100 pour nos tests), on calcul les statistiques associées à chaque algorithme.

L'énumération ne nécessite pas d'itérations car elles fournis toujours les même résultats, la ou les meilleurs solutions.

Voici nos résultats:

\subsection{Algorithme d'énumération}
\subsubsection{4 Sommets en 2 classes, mu = [1; 0.5; 0]}
\begin{verbatim}
-------ENUMERATE-------
nbS = 4; nbK = 2; delta_max = 3; mu = 1.0
best score is : 1,
 total time : 0.00038140779361128807
-------ENUMERATE-------
nbS = 4; nbK = 2; delta_max = 3; mu = 0.5
best score is : 1,
 total time : 0.00037381192669272423
-------ENUMERATE-------
nbS = 4; nbK = 2; delta_max = 3; mu = 0.0
best score is : 1,
 total time : 0.0003458610735833645
\end{verbatim}
\subsubsection{5 Sommets en 2 classes, mu = [1; 0.5; 0]}
\begin{verbatim}
-------ENUMERATE-------
nbS = 5; nbK = 2; delta_max = 3; mu = 1.0
best score is : 3,
 total time : 0.0008135139942169189
-------ENUMERATE-------
nbS = 5; nbK = 2; delta_max = 3; mu = 0.5
best score is : 3,
 total time : 0.0008682990446686745
-------ENUMERATE-------
nbS = 5; nbK = 2; delta_max = 3; mu = 0.0
best score is : 3,
 total time : 0.0007814168930053711
\end{verbatim}
\subsubsection{20 Sommets en 2 classes, mu = [1; 0.5; 0]}
\begin{verbatim}
-------ENUMERATE-------
nbS = 20; nbK = 2; delta_max = 3; mu = 1.0
best score is : 39,
 total time : 163.1898413640447
-------ENUMERATE-------
nbS = 20; nbK = 2; delta_max = 3; mu = 0.5
best score is : 39,
 total time : 160.39260955667123
-------ENUMERATE-------
nbS = 20; nbK = 2; delta_max = 3; mu = 0.0
best score is : 39,
 total time : 164.47484961291775
\end{verbatim}

\subsection{Descente de Gradient}
\subsubsection{5 Sommets en 2 classes, mu = [1; 0.5; 0]}
\begin{verbatim}
-------MULTI_PROC HILLCLIMB-------
Running on 16 proc
nbS = 5; nbK = 2; delta_max = 3; mu = 1.0; move_operator= pick_gen
for 100 iteration with 100 max_evaluations each, 
 total time in sec : 0.10310757299885154
 best score found is 3,
 mean time in sec : 0.001071562976576388,
 mean best_score : 3.63, EcT : 0.1714466079977652
 mean num_eval : 7.07
-------MULTI_PROC HILLCLIMB-------
Running on 16 proc
nbS = 5; nbK = 2; delta_max = 3; mu = 0.5; move_operator= pick_gen
for 100 iteration with 100 max_evaluations each, 
 total time in sec : 0.10206909896805882
 best score found is 3,
 mean time in sec : 0.0012981305923312903,
 mean best_score : 3.36, EcT : 0.23868325657594203
 mean num_eval : 6.82
-------MULTI_PROC HILLCLIMB-------
Running on 16 proc
nbS = 5; nbK = 2; delta_max = 3; mu = 0.0; move_operator= pick_gen
for 100 iteration with 100 max_evaluations each, 
 total time in sec : 0.10367809794843197
 best score found is 3,
 mean time in sec : 0.0010838805558159947,
 mean best_score : 3.05, EcT : 0.2190429135575903
 mean num_eval : 6.94
\end{verbatim}
\subsubsection{15 Sommets en 3 classes, mu = [1; 0.5; 0]}
\begin{verbatim}
-------MULTI_PROC HILLCLIMB-------
Running on 16 proc
nbS = 15; nbK = 3; delta_max = 3; mu = 1.0; move_operator= pick_gen
for 100 iteration with 100 max_evaluations each, 
 total time in sec : 0.2024659407325089
 best score found is 58,
 mean time in sec : 0.023689089985564352,
 mean best_score : 61.010000000000005, EcT : 1.1520293907586374
 mean num_eval : 21.36
-------MULTI_PROC HILLCLIMB-------
Running on 16 proc
nbS = 15; nbK = 3; delta_max = 3; mu = 0.5; move_operator= pick_gen
for 100 iteration with 100 max_evaluations each, 
 total time in sec : 0.20349178183823824
 best score found is 58,
 mean time in sec : 0.024920869143679737,
 mean best_score : 61.01, EcT : 1.2956414892580694
 mean num_eval : 21.34
-------MULTI_PROC HILLCLIMB-------
Running on 16 proc
nbS = 15; nbK = 3; delta_max = 3; mu = 0.0; move_operator= pick_gen
for 100 iteration with 100 max_evaluations each, 
 total time in sec : 0.2024315781891346
 best score found is 59,
 mean time in sec : 0.020131065961904823,
 mean best_score : 61.58, EcT : 1.464599093626138
 mean num_eval : 18.2
\end{verbatim}
\subsubsection{20 Sommets en 3 classes, mu = [1; 0.5; 0]}
\begin{verbatim}
-------MULTI_PROC HILLCLIMB-------
Running on 16 proc
nbS = 20; nbK = 3; delta_max = 3; mu = 1.0; move_operator= pick_gen
for 100 iteration with 100 max_evaluations each, 
 total time in sec : 0.9040888692252338
 best score found is 53,
 mean time in sec : 0.11574626796413214,
 mean best_score : 56.768, EcT : 1.9256052439572229
 mean num_eval : 90.18
-------MULTI_PROC HILLCLIMB-------
Running on 16 proc
nbS = 20; nbK = 3; delta_max = 3; mu = 0.5; move_operator= pick_gen
for 100 iteration with 100 max_evaluations each, 
 total time in sec : 0.8043653820641339
 best score found is 53,
 mean time in sec : 0.11241874734871089,
 mean best_score : 56.617000000000004, EcT : 1.7737026842515662
 mean num_eval : 90.2
-------MULTI_PROC HILLCLIMB-------
Running on 16 proc
nbS = 20; nbK = 3; delta_max = 3; mu = 0.0; move_operator= pick_gen
for 100 iteration with 100 max_evaluations each, 
 total time in sec : 0.8047608300112188
 best score found is 53,
 mean time in sec : 0.09706705980934202,
 mean best_score : 57.68, EcT : 2.3989896863370452
 mean num_eval : 80.08
\end{verbatim}
\subsubsection{50 Sommets en 3 classes, mu = [1; 0.5; 0]}
\begin{verbatim}
-------MULTI_PROC HILLCLIMB-------
Running on 16 proc
nbS = 50; nbK = 3; delta_max = 3; mu = 1.0; move_operator= pick_gen
for 100 iteration with 100 max_evaluations each, 
 total time in sec : 14.029909709934145
 best score found is 288,
 mean time in sec : 2.0499937719944863,
 mean best_score : 310.6014, EcT : 7.05828047563479
 mean num_eval : 100
-------MULTI_PROC HILLCLIMB-------
Running on 16 proc
nbS = 50; nbK = 3; delta_max = 3; mu = 0.5; move_operator= pick_gen
for 100 iteration with 100 max_evaluations each, 
 total time in sec : 14.246775717940181
 best score found is 292,
 mean time in sec : 2.059209058196284,
 mean best_score : 312.4839, EcT : 6.499423112706218
 mean num_eval : 100
-------MULTI_PROC HILLCLIMB-------
Running on 16 proc
nbS = 50; nbK = 3; delta_max = 3; mu = 0.0; move_operator= pick_gen
for 100 iteration with 100 max_evaluations each, 
 total time in sec : 13.881034784018993
 best score found is 294,
 mean time in sec : 2.0022386860102417,
 mean best_score : 310.4, EcT : 6.935489175322106
 mean num_eval : 100
\end{verbatim}
\subsubsection{100 Sommets en 4 classes, mu = [1; 0.5; 0]}
\begin{verbatim}
-------MULTI_PROC HILLCLIMB-------
Running on 16 proc
nbS = 100; nbK = 4; delta_max = 3; mu = 1.0; move_operator= pick_gen
for 100 iteration with 100 max_evaluations each, 
 total time in sec : 275.9682481228374
 best score found is 1430,
 mean time in sec : 40.02110815170687,
 mean best_score : 1476.0958, EcT : 14.049210494371259
 mean num_eval : 100
-------MULTI_PROC HILLCLIMB-------
Running on 16 proc
nbS = 100; nbK = 4; delta_max = 3; mu = 0.5; move_operator= pick_gen
for 100 iteration with 100 max_evaluations each, 
 total time in sec : 278.2741830400191
 best score found is 1451,
 mean time in sec : 40.016724326196126,
 mean best_score : 1475.13245, EcT : 12.389276001686275
 mean num_eval : 100
-------MULTI_PROC HILLCLIMB-------
Running on 16 proc
nbS = 100; nbK = 4; delta_max = 3; mu = 0.0; move_operator= pick_gen
for 100 iteration with 100 max_evaluations each, 
 total time in sec : 274.8639584830962
 best score found is 1444,
 mean time in sec : 39.720520393801856,
 mean best_score : 1472.87, EcT : 13.68967390835778
 mean num_eval : 100
\end{verbatim}

\subsection{Recuit Simulé}
\subsubsection{5 Sommets en 2 classes, mu = [1; 0.5; 0]}
\begin{verbatim}
-------MULTI_PROC SIMULATED ANNEALING-------
Running on 16 proc
nbS = 5; nbK = 2; delta_max = 3; mu = 1.0;
final_temp = 0.89242491504672; alpha = 0.95; move_operator= pick_gen
for 100 iteration with 100 max_evaluations each, 
 best score found is 3,
 total time in sec : 0.15890276758000255
 mean time in sec : 0.019896002300083638,
 mean best_score : 3.6, EcT : 0.0
 mean num_eval : 100,
 mean end temperature : 1.0925721940968705
-------MULTI_PROC SIMULATED ANNEALING-------
Running on 16 proc
nbS = 5; nbK = 2; delta_max = 3; mu = 0.5;
final_temp = 1.2779281874799289; alpha = 0.95; move_operator= pick_gen
for 100 iteration with 100 max_evaluations each, 
 best score found is 3,
 total time in sec : 0.14763296116143465
 mean time in sec : 0.019895954332314433,
 mean best_score : 3.3, EcT : 0.0
 mean num_eval : 100,
 mean end temperature : 1.1712864838845436
-------MULTI_PROC SIMULATED ANNEALING-------
Running on 16 proc
nbS = 5; nbK = 2; delta_max = 3; mu = 0.0;
final_temp = 1.3451875657683463; alpha = 0.95; move_operator= pick_gen
for 100 iteration with 100 max_evaluations each, 
 best score found is 3,
 total time in sec : 0.135501594748348
 mean time in sec : 0.018298799749463798,
 mean best_score : 3.0, EcT : 0.0
 mean num_eval : 100,
 mean end temperature : 1.2735855241323177
\end{verbatim}
\subsubsection{15 Sommets en 3 classes, mu = [1; 0.5; 0]}
\begin{verbatim}
-------MULTI_PROC SIMULATED ANNEALING-------
Running on 16 proc
nbS = 15; nbK = 3; delta_max = 3; mu = 1.0;
final_temp = 1.2779281874799289; alpha = 0.95; move_operator= pick_gen
for 100 iteration with 100 max_evaluations each, 
 best score found is 58,
 total time in sec : 1.479973640292883
 mean time in sec : 0.21311060020234435,
 mean best_score : 60.56866666666667, EcT : 1.3138118050294691
 mean num_eval : 100,
 mean end temperature : 1.3772049646765163
-------MULTI_PROC SIMULATED ANNEALING-------
Running on 16 proc
nbS = 15; nbK = 3; delta_max = 3; mu = 0.5;
final_temp = 1.3451875657683463; alpha = 0.95; move_operator= pick_gen
for 100 iteration with 100 max_evaluations each, 
 best score found is 58,
 total time in sec : 1.5027663107030094
 mean time in sec : 0.21560919662937522,
 mean best_score : 60.43933333333334, EcT : 1.4292077498978677
 mean num_eval : 100,
 mean end temperature : 1.4315769926409938
-------MULTI_PROC SIMULATED ANNEALING-------
Running on 16 proc
nbS = 15; nbK = 3; delta_max = 3; mu = 0.0;
final_temp = 1.4159869113351014; alpha = 0.95; move_operator= pick_gen
for 100 iteration with 100 max_evaluations each, 
 best score found is 58,
 total time in sec : 1.4449690897017717
 mean time in sec : 0.20822494091000407,
 mean best_score : 60.05, EcT : 1.3361712223190316
 mean num_eval : 100,
 mean end temperature : 1.494534520905948
\end{verbatim}
\subsubsection{20 Sommets en 3 classes, mu = [1; 0.5; 0]}
\begin{verbatim}
-------MULTI_PROC SIMULATED ANNEALING-------
Running on 16 proc
nbS = 20; nbK = 3; delta_max = 3; mu = 1.0;
final_temp = 1.2779281874799289; alpha = 0.95; move_operator= pick_gen
for 100 iteration with 100 max_evaluations each, 
 best score found is 58,
 total time in sec : 2.100551615934819
 mean time in sec : 0.30202195783145724,
 mean best_score : 63.7975, EcT : 2.5055027822613165
 mean num_eval : 100,
 mean end temperature : 1.313465545276881
-------MULTI_PROC SIMULATED ANNEALING-------
Running on 16 proc
nbS = 20; nbK = 3; delta_max = 3; mu = 0.5;
final_temp = 1.5689605665762896; alpha = 0.95; move_operator= pick_gen
for 100 iteration with 100 max_evaluations each, 
 best score found is 56,
 total time in sec : 2.204790683928877
 mean time in sec : 0.32301434411667285,
 mean best_score : 63.796, EcT : 2.8453504515119383
 mean num_eval : 100,
 mean end temperature : 1.3380972607900203
-------MULTI_PROC SIMULATED ANNEALING-------
Running on 16 proc
nbS = 20; nbK = 3; delta_max = 3; mu = 0.0;
final_temp = 1.6515374385013575; alpha = 0.95; move_operator= pick_gen
for 100 iteration with 100 max_evaluations each, 
 best score found is 56,
 total time in sec : 2.155007839668542
 mean time in sec : 0.3094922862760723,
 mean best_score : 63.9, EcT : 2.8762349126466136
 mean num_eval : 100,
 mean end temperature : 1.3314336387531318
\end{verbatim}
\subsubsection{50 Sommets en 3 classes, mu = [1; 0.5; 0]}
\begin{verbatim}
-------MULTI_PROC SIMULATED ANNEALING-------
Running on 16 proc
nbS = 50; nbK = 3; delta_max = 3; mu = 1.0;
final_temp = 2.2467088258818433; alpha = 0.95; move_operator= pick_gen
for 100 iteration with 100 max_evaluations each, 
 best score found is 304,
 total time in sec : 19.639723698608577
 mean time in sec : 2.8562343602767215,
 mean best_score : 327.0544, EcT : 7.9476169250838185
 mean num_eval : 100,
 mean end temperature : 1.5985980839004053
-------MULTI_PROC SIMULATED ANNEALING-------
Running on 16 proc
nbS = 50; nbK = 3; delta_max = 3; mu = 0.5;
final_temp = 1.8299583806109228; alpha = 0.95; move_operator= pick_gen
for 100 iteration with 100 max_evaluations each, 
 best score found is 312,
 total time in sec : 19.559158163145185
 mean time in sec : 2.8592799879610538,
 mean best_score : 327.18489999999997, EcT : 6.5545652985171525
 mean num_eval : 100,
 mean end temperature : 1.562941702556063
-------MULTI_PROC SIMULATED ANNEALING-------
Running on 16 proc
nbS = 50; nbK = 3; delta_max = 3; mu = 0.0;
final_temp = 1.5689605665762896; alpha = 0.95; move_operator= pick_gen
for 100 iteration with 100 max_evaluations each, 
 best score found is 314,
 total time in sec : 19.80668053869158
 mean time in sec : 2.875196288712323,
 mean best_score : 326.67, EcT : 6.72107344721275
 mean num_eval : 100,
 mean end temperature : 1.598711134190034
\end{verbatim}
\subsubsection{100 Sommets en 4 classes, mu = [1; 0.5; 0]}
\begin{verbatim}
-------MULTI_PROC SIMULATED ANNEALING-------
Running on 16 proc
nbS = 100; nbK = 4; delta_max = 3; mu = 1.0;
final_temp = 1.926271979590445; alpha = 0.95; move_operator= pick_gen
for 100 iteration with 100 max_evaluations each, 
 best score found is 1463,
 total time in sec : 306.5104563161731
 mean time in sec : 44.28779415917583,
 mean best_score : 1497.5357999999999, EcT : 14.775266268089023
 mean num_eval : 100,
 mean end temperature : 1.8711562858658632
-------MULTI_PROC SIMULATED ANNEALING-------
Running on 16 proc
nbS = 100; nbK = 4; delta_max = 3; mu = 0.5;
final_temp = 1.8299583806109228; alpha = 0.95; move_operator= pick_gen
for 100 iteration with 100 max_evaluations each, 
 best score found is 1463,
 total time in sec : 306.9560006177053
 mean time in sec : 44.25927069529891,
 mean best_score : 1495.07265, EcT : 13.933485200256166
 mean num_eval : 100,
 mean end temperature : 1.8438790330748873
-------MULTI_PROC SIMULATED ANNEALING-------
Running on 16 proc
nbS = 100; nbK = 4; delta_max = 3; mu = 0.0;
final_temp = 1.8299583806109228; alpha = 0.95; move_operator= pick_gen
for 100 iteration with 100 max_evaluations each, 
 best score found is 1462,
 total time in sec : 305.0477506122552
 mean time in sec : 44.03396145147271,
 mean best_score : 1497.22, EcT : 14.388954798803846
 mean num_eval : 100,
 mean end temperature : 1.8231735690116595
\end{verbatim}

\subsection{Méthode Tabou}
\subsubsection{5 Sommets en 2 classes, mu = [1; 0.5; 0]}
\begin{verbatim}
-------MULTI_PROC TABUSEARCH-------
Running on 16 proc
nbS = 5; nbK = 2; delta_max = 3; mu = 1.0; move_operator= pick_gen
for 100 iteration with 100 max_evaluations each, 
 total time in sec : 0.017625250853598118
 best score found is 3,
 mean time in sec : 0.0014280724758282304,
 mean best_score : 3.72, EcT : 0.32659863237109027
 mean num_eval : 2.28
-------MULTI_PROC TABUSEARCH-------
Running on 16 proc
nbS = 5; nbK = 2; delta_max = 3; mu = 0.5; move_operator= pick_gen
for 100 iteration with 100 max_evaluations each, 
 total time in sec : 0.020844256971031427
 best score found is 3,
 mean time in sec : 0.0016844153171405197,
 mean best_score : 3.3099999999999996, EcT : 0.1
 mean num_eval : 2.28
-------MULTI_PROC TABUSEARCH-------
Running on 16 proc
nbS = 5; nbK = 2; delta_max = 3; mu = 0.0; move_operator= pick_gen
for 100 iteration with 100 max_evaluations each, 
 total time in sec : 0.016749583650380373
 best score found is 3,
 mean time in sec : 0.0011243263492360712,
 mean best_score : 3.05, EcT : 0.2190429135575903
 mean num_eval : 2.34
\end{verbatim}
\subsubsection{15 Sommets en 3 classes, mu = [1; 0.5; 0]}
\begin{verbatim}
-------MULTI_PROC TABUSEARCH-------
Running on 16 proc
nbS = 15; nbK = 3; delta_max = 3; mu = 1.0; move_operator= pick_gen
for 100 iteration with 100 max_evaluations each, 
 total time in sec : 0.15875420486554503
 best score found is 58,
 mean time in sec : 0.021858335416764022,
 mean best_score : 61.25, EcT : 1.0384039812533599
 mean num_eval : 3.3
-------MULTI_PROC TABUSEARCH-------
Running on 16 proc
nbS = 15; nbK = 3; delta_max = 3; mu = 0.5; move_operator= pick_gen
for 100 iteration with 100 max_evaluations each, 
 total time in sec : 0.18269887520000339
 best score found is 59,
 mean time in sec : 0.022573766084387898,
 mean best_score : 60.910000000000004, EcT : 1.0415121097459572
 mean num_eval : 3.37
-------MULTI_PROC TABUSEARCH-------
Running on 16 proc
nbS = 15; nbK = 3; delta_max = 3; mu = 0.0; move_operator= pick_gen
for 100 iteration with 100 max_evaluations each, 
 total time in sec : 0.15549310809001327
 best score found is 58,
 mean time in sec : 0.019807322970591484,
 mean best_score : 61.51, EcT : 1.534090722035155
 mean num_eval : 2.78
\end{verbatim}
\subsubsection{20 Sommets en 3 classes, mu = [1; 0.5; 0]}
\begin{verbatim}
-------MULTI_PROC TABUSEARCH-------
Running on 16 proc
nbS = 20; nbK = 3; delta_max = 3; mu = 1.0; move_operator= pick_gen
for 100 iteration with 100 max_evaluations each, 
 total time in sec : 0.9273185259662569
 best score found is 53,
 mean time in sec : 0.13213871700689198,
 mean best_score : 55.864, EcT : 1.6160664302777619
 mean num_eval : 13
-------MULTI_PROC TABUSEARCH-------
Running on 16 proc
nbS = 20; nbK = 3; delta_max = 3; mu = 0.5; move_operator= pick_gen
for 100 iteration with 100 max_evaluations each, 
 total time in sec : 1.029785099439323
 best score found is 53,
 mean time in sec : 0.1434148733271286,
 mean best_score : 55.523, EcT : 1.3498394966316054
 mean num_eval : 13.32
-------MULTI_PROC TABUSEARCH-------
Running on 16 proc
nbS = 20; nbK = 3; delta_max = 3; mu = 0.0; move_operator= pick_gen
for 100 iteration with 100 max_evaluations each, 
 total time in sec : 0.761064148042351
 best score found is 53,
 mean time in sec : 0.10573584100231528,
 mean best_score : 57.6, EcT : 2.4329389468689753
 mean num_eval : 9.31
\end{verbatim}
\subsubsection{50 Sommets en 3 classes, mu = [1; 0.5; 0]}
\begin{verbatim}
-------MULTI_PROC TABUSEARCH-------
Running on 16 proc
nbS = 50; nbK = 3; delta_max = 3; mu = 1.0; move_operator= pick_gen
for 100 iteration with 100 max_evaluations each, 
 total time in sec : 98.10643908800557
 best score found is 266,
 mean time in sec : 14.21196602887474,
 mean best_score : 276.8064, EcT : 5.5716967874166565
 mean num_eval : 44.07
-------MULTI_PROC TABUSEARCH-------
Running on 16 proc
nbS = 50; nbK = 3; delta_max = 3; mu = 0.5; move_operator= pick_gen
for 100 iteration with 100 max_evaluations each, 
 total time in sec : 104.85385024594143
 best score found is 267,
 mean time in sec : 14.863117454447783,
 mean best_score : 275.96779999999995, EcT : 4.811031217289839
 mean num_eval : 44.91
-------MULTI_PROC TABUSEARCH-------
Running on 16 proc
nbS = 50; nbK = 3; delta_max = 3; mu = 0.0; move_operator= pick_gen
for 100 iteration with 100 max_evaluations each, 
 total time in sec : 67.86398608796299
 best score found is 268,
 mean time in sec : 9.82464586332906,
 mean best_score : 281.76, EcT : 5.747410484098666
 mean num_eval : 30.92
\end{verbatim}
\subsubsection{100 Sommets en 3 classes, mu = [1; 0.5; 0]}
\begin{verbatim}
-------MULTI_PROC TABUSEARCH-------
Running on 16 proc
nbS = 100; nbK = 3; delta_max = 3; mu = 1.0; move_operator= pick_gen; tabu_maxsize = 15
for 100 iteration with 100 max_evaluations each,
 total time in sec : 4507.274052148219
 best score found is 1146,
 mean time in sec : 661.1381062711449,
 mean best_score : 1169.6046999999999, EcT : 12.228225459521465
 mean num_eval : 94.99
-------MULTI_PROC TABUSEARCH-------
Running on 16 proc
nbS = 100; nbK = 3; delta_max = 3; mu = 0.5; move_operator= pick_gen; tabu_maxsize = 15
for 100 iteration with 100 max_evaluations each,
 total time in sec : 4468.009353993926
 best score found is 1146,
 mean time in sec : 652.4391142716352,
 mean best_score : 1167.4124000000002, EcT : 11.288820483965184
 mean num_eval : 95.13
-------MULTI_PROC TABUSEARCH-------
Running on 16 proc
nbS = 100; nbK = 3; delta_max = 3; mu = 0.0; move_operator= pick_gen; tabu_maxsize = 15
for 100 iteration with 100 max_evaluations each,
 total time in sec : 4115.974301184062
 best score found is 1142,
 mean time in sec : 597.1457226081518,
 mean best_score : 1168.06, EcT : 10.223581376485257
 mean num_eval : 79.76
\end{verbatim}

\section{Comparaison des résultats}

\subsection{Algorithme d'énumération}
L'algorithme d'énumération fournis systématiquement la ou les solutions optimales, cependant le temps d’exécution augmente exponentiellement en fonction du nombre de classe et de sommets à traiter. C'est pour cette raison que nous nous sommes arrêté à partir de 20 sommets en 3 classes.\\\\
5 Sommets en 2 classes : 0.0007814168930053711 secondes\\
20 Sommets en 2 classes : 160.39260955667123 secondes

\subsection{Descente de Gradient}

\subsubsection{Interprétations des résultats}
Les résultats obtenus sont des moyennes de valeur trouvé par les itérations successive de l'algorithme.\\
\begin{enumerate}
	\item{ect}: l'écart type entre les solutions
	\item{best\_score}: meilleur score associé à une solution obtenu
	\item{time}: temps en secondes
	\item{num\_eval}: le nombre d'évaluations de l'algorithme avant de remplir une condition d'arrêt, de 0 à 100.
\end{enumerate}

\subsubsection{Comparaison avec l'énumération}
L'algorithme de descente de gradient nous permet de trouver des optimums locaux, en itérant cette algorithme avec une solution de départ aléatoire nous augmentons les chances de trouver l'optimum global, c'est pour cela que, sur les petits graphes, le meilleur score correspond au résultats de l'algorithme d'énumération.\\\\
Énumération: 20 Sommets en 2 classes : 160 secondes\\
Descente: 20 Sommets en 2 classes 100 itérations : 0.40 seconde\\

Le score trouvé étant identique, l'algorithme de descente de gradient ce révèle plus performant, même si le résultat n'est pas forcément la solution optimale, le gain de temps n'est pas négligeable.\\

\subsection{Recuit Simulé}

\subsection{Méthode Tabou}

\section{Limites}

\section{Conclusion}

\end{document}